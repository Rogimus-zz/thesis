
\chapter{Background}
\label{chapter1}
Eilenberg and Steenrod inspired the following
axiomatic definition of a cohomology
theory, intending to capture the
nice properties
of ordinary cohomology  [123123].
\begin{defn}
  An ordinary cohomology theory is a sequence of
  functors
  $E^i$ for $i\geq 0$ mapping a pair of
  spaces $(X,A)$ to an abelian group
  $E^i(X,A)$,
  as well as natural maps
  $\delta_i: E^i(A,\emptyset)
  \to E^{i+1}(X,A)$ satisfying
  the following axioms:
  \begin{enumerate}
  \item
    given two homotopic maps
    $f\simeq g:(X,A)\to (Y,B)$
    the induced maps on cohomology groups
    is the same
  \item
    To any pair $(X,A)$ there is
    an associated long exact sequence:
    %%
    \[
      \small
    \begin{tikzcd}
      \cdots \arrow[r]& E^{i}(X,A)\arrow[r,"q_*"]&
      E^{i}(X,\emptyset) \arrow[r,"i_*"]& E^i(A
      ,\emptyset)
      \arrow[r,"\delta_i"] &E^{i+1}(X,A)\arrow[r]&\cdots
    \end{tikzcd}
    \]
    %%
    where $i_*$ and $q_*$ are the maps induced
    by the inclusions $i: (A,\emptyset) \to
    (X,\emptyset)$ and $q: (X,\emptyset)\to (X,A)$
  \item 
    (\textit{Excision})
    Given any subspace $U\subset A \subset X$ then
    the inclusion of \\$(X\setminus U, A\setminus U)
    $ into $(X,A)$ induces an isomorphism on cohomology
    groups
    \[
      \begin{tikzcd}
        E^i(X,A)\arrow[r, "i_*"] & E^i(X\setminus U,
        A\setminus U)
      \end{tikzcd}
    \]
  \item (\textit{Additivity})
    If $(X,A)$ is the disjoint union of subsets
    $\{ (X_j,A_j) \}_{j\in J}$
    then we get the following isomorphism for all
    $i$:
    \[
      \begin{tikzcd}
        E^i(X,A)\arrow[r, "\sim"] & \prod_{j\in J}
        E^i(X_j,A_j)
      \end{tikzcd}
    \]
  \item
    (\textit{dimension}) $E^i(pt)=0$ for all $i>0$.
  \end{enumerate}
\end{defn}
Let us drop the cumbersome notation
$E^i(X,\emptyset)$ and simply write $E^i(X)$
instead. 

Unfortunately this definition
(in particular the dimension axiom)
is quite restrictive and it excludes many
nice functors we might like to think of
as cohomology theories, such as
topological K-theory which
has $K^{2i}(pt)\cong K^{0}(pt)\cong \mathbb{Z}$ for all
$i\in \mathbb{Z}$. The next definition
tries to rectify this problem:

\begin{defn}
  A \textit{generalised cohomology theory} is
  a sequence of functors $E^i$ with maps
  $\delta_i$ all as above which satisfies
  axioms 1-4 of \textit{Definition 1.1}.
\end{defn}

Let us now consider a particular set of
generalised cohomology theories with
rather nice nice properties.

\begin{defn}
  A generalised cohomology theory $E$
  is called an \textit{even periodic ring
    theory} if it
  satisfies the following properties:
  \begin{enumerate}
  \item $E^*(pt)$ is a commutative ring
  \item $E^i(pt) = 0$ whenever $i$ is odd
  \item there exists elements
    $\beta\in E^{2}(pt)$ and $\beta^{-1}
    \in E^{-2}(pt)$ such that $\beta\cdot \beta^{-1}
    =1$ in $E^*(pt)$
  \end{enumerate}
\end{defn}

The obvious example is once again K-theory.
Indeed the properties in
\textit{Definition 1.3} follow since
$K^*(pt)\cong\mathbb{Z}
[\alpha,\alpha^{-1}]$
where 